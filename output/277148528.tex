% Options for packages loaded elsewhere
\PassOptionsToPackage{unicode}{hyperref}
\PassOptionsToPackage{hyphens}{url}
%
\documentclass[
  11pt,
  a4paper,
  oneside]{article}
\usepackage{amsmath,amssymb}
\usepackage{iftex}
\ifPDFTeX
  \usepackage[T1]{fontenc}
  \usepackage[utf8]{inputenc}
  \usepackage{textcomp} % provide euro and other symbols
\else % if luatex or xetex
  \usepackage{unicode-math} % this also loads fontspec
  \defaultfontfeatures{Scale=MatchLowercase}
  \defaultfontfeatures[\rmfamily]{Ligatures=TeX,Scale=1}
\fi
\usepackage{lmodern}
\ifPDFTeX\else
  % xetex/luatex font selection
  \setmainfont[]{helvet}
\fi
% Use upquote if available, for straight quotes in verbatim environments
\IfFileExists{upquote.sty}{\usepackage{upquote}}{}
\IfFileExists{microtype.sty}{% use microtype if available
  \usepackage[]{microtype}
  \UseMicrotypeSet[protrusion]{basicmath} % disable protrusion for tt fonts
}{}
\makeatletter
\@ifundefined{KOMAClassName}{% if non-KOMA class
  \IfFileExists{parskip.sty}{%
    \usepackage{parskip}
  }{% else
    \setlength{\parindent}{0pt}
    \setlength{\parskip}{6pt plus 2pt minus 1pt}}
}{% if KOMA class
  \KOMAoptions{parskip=half}}
\makeatother
\usepackage{xcolor}
\usepackage[top=2cm,left=2.5cm,right=2.5cm,bottom=3.5cm]{geometry}
\usepackage{graphicx}
\makeatletter
\def\maxwidth{\ifdim\Gin@nat@width>\linewidth\linewidth\else\Gin@nat@width\fi}
\def\maxheight{\ifdim\Gin@nat@height>\textheight\textheight\else\Gin@nat@height\fi}
\makeatother
% Scale images if necessary, so that they will not overflow the page
% margins by default, and it is still possible to overwrite the defaults
% using explicit options in \includegraphics[width, height, ...]{}
\setkeys{Gin}{width=\maxwidth,height=\maxheight,keepaspectratio}
% Set default figure placement to htbp
\makeatletter
\def\fps@figure{htbp}
\makeatother
\setlength{\emergencystretch}{3em} % prevent overfull lines
\providecommand{\tightlist}{%
  \setlength{\itemsep}{0pt}\setlength{\parskip}{0pt}}
\setcounter{secnumdepth}{5}
\ifLuaTeX
\usepackage[bidi=basic]{babel}
\else
\usepackage[bidi=default]{babel}
\fi
\babelprovide[main,import]{spanish}
\ifPDFTeX
\else
\babelfont{rm}[]{helvet}
\fi
% get rid of language-specific shorthands (see #6817):
\let\LanguageShortHands\languageshorthands
\def\languageshorthands#1{}
\usepackage{lscape}
\newcommand{\blandscape}{\begin{landscape}}
\newcommand{\elandscape}{\end{landscape}}
\newcommand{\Rlogo}{\protect\includegraphics[height=1.8ex,keepaspectratio]{images/Rlogo.png}}
\usepackage{floatrow}
\floatplacement{figure}{H}
\floatplacement{table}{H}
\floatsetup[figure]{capposition=top}
\floatsetup[table]{capposition=top}
\addtolength{\skip\footins}{1pc plus 2pt}
\usepackage{titlesec}
\titleformat*{\section}{\filright \normalsize \bfseries}
\titleformat*{\subsection}{\filright \normalsize\bfseries}
\titleformat*{\subsubsection}{\filright \normalsize\bfseries}
\renewcommand{\thesection}{\Roman{section}.}
\renewcommand{\thesubsection}{\Alph{subsection}.}
\renewcommand{\thesubsubsection}{\thesubsection \arabic{subsubsection}.}
\usepackage{helvet}
\renewcommand{\familydefault}{\sfdefault}
\usepackage{colortbl}
\usepackage{array}
\newcolumntype{M}[1]{>{\centering\arraybackslash}m{#1}}
\usepackage{graphicx}
\usepackage{hhline,colortbl}
\usepackage{fancyhdr}
\usepackage{longtable}
\pagestyle{fancy}
\setlength{\headheight}{71pt}
\addtolength{\topmargin}{-4pt}
\fancyhf{}
\renewcommand{\headrulewidth}{0pt}
\fancyhead[L]{ \setlength{\arrayrulewidth}{0.35mm} \arrayrulecolor{white} \begin{tabular} { | >{\centering\arraybackslash}m{1.5cm} | >{\centering\arraybackslash}m{1.5cm} | >{\centering\arraybackslash}m{3cm} | >{\centering\arraybackslash}m{5cm} | >{\centering\arraybackslash}m{2.5cm} |} \includegraphics[width=1cm]{images/peru} & \cellcolor{red} \textcolor{white}{PERÚ} & \cellcolor[gray]{0.2} \scriptsize \textcolor{white}{Presidencia del Consejo de Ministros} & \cellcolor[gray]{0.4} \scriptsize \textcolor{white}{Instituto Nacional de Defensa de la Competencia y de la Protección de la Propiedad Intelectual} &  \cellcolor[gray]{0.5} \scriptsize \textcolor{white}{Dirección de Fiscalización} \\ \end{tabular}\\\vspace{5mm}}
\fancyhead[C] {{\scriptsize "Decenio de la igualdad de oportunidades para mujeres y hombres"}\\ {\scriptsize "Año del Bicentenario del Perú{:} 200 años de independencia"}}
\fancyfoot[C]{\thepage}
\ifLuaTeX
  \usepackage{selnolig}  % disable illegal ligatures
\fi
\IfFileExists{bookmark.sty}{\usepackage{bookmark}}{\usepackage{hyperref}}
\IfFileExists{xurl.sty}{\usepackage{xurl}}{} % add URL line breaks if available
\urlstyle{same}
\hypersetup{
  pdftitle={DFI - Ficha de evaluación},
  pdfauthor={DFI},
  pdflang={es},
  pdfkeywords={informe, r studio, r markdown},
  hidelinks,
  pdfcreator={LaTeX via pandoc}}

\author{}
\date{\vspace{-2.5em}}

\begin{document}

\centerline{\textbf{FICHA DE EVALUACIÓN \protect\includegraphics[height=1.8ex,keepaspectratio]{images/Rlogo.png}}}

\setlength{\arrayrulewidth}{0.35mm} 
\arrayrulecolor{black} 
\begin{table}
\setlength\extrarowheight{5pt}
\begin{tabular}
  { 
  | M{3.5cm} 
  | M{3.5cm} 
  | M{3.5cm}
  | M{3.5cm}
  |} 
  \hline 
  \textbf{Número de ficha} & 277148528 & \textbf{Fecha de elaboración} & 11-10-2023 \\ [10pt] 
  \hline 
  \end{tabular}
\end{table}

\section{APROBACIÓN}\label{aprobaciuxf3n}

\begin{table}
\setlength{\arrayrulewidth}{0.35mm} 
\arrayrulecolor{black} 
\setlength\extrarowheight{10pt}
\begin{tabular}
  { 
  | M{3.5cm} 
  | M{3.5cm} 
  | M{7.5cm}
  |} 
  \hline 
  \textbf{Nombres y apellidos} & \textbf{Cargo} & \textbf{Firma} \\ [20pt]
  \hline 
  DDD & NA & \\ [20pt]
  \hline 
  Kattya del Rosario Palacios Herrera & Jefa de Equipo Legal &  \\ [20pt]
  \hline 
  Mariajosé Calle Merino-Carrasco & Coordinadora de disuasión &  \\ [20pt]
  \hline 
  Dante Ramón Guerrero Barreto & Coordinador General &  \\ [20pt]
  \hline 
  Milagros Cecilia Pozo Ascuña & Directora &  \\ [20pt]
  \hline
  \end{tabular}
\end{table}
\vspace{5mm}

\section{DATOS DEL REPORTE}\label{datos-del-reporte}

\setlength{\arrayrulewidth}{0.35mm} 
\arrayrulecolor{black} 
\begin{table}
\setlength\extrarowheight{5pt}
\begin{tabular}
  { 
  | M{3.5cm} 
  | M{3.5cm} 
  | M{3.5cm}
  | M{3.5cm}
  |} 
  \hline 
  \textbf{Número del registro} & R-200-2023 & \textbf{Fecha del reporte} & 04-10-2023 \\ [10pt]
  \hline 
  \end{tabular}
\end{table}

\setlength{\arrayrulewidth}{0.35mm} 
\arrayrulecolor{black} 
\begin{table}
\setlength\extrarowheight{3pt}
\begin{tabular}
  { 
  | M{7.5cm} 
  | M{7.5cm} 
  |} 
  \hline 
  \textbf{Autor del reporte} & \textbf{Medio de recepción}  \\ [10pt]
  \hline 
   Redes sociales & Correo electrónico \\ [10pt]
  \hline
  \end{tabular}
\end{table}

\setlength{\arrayrulewidth}{0.35mm} 
\arrayrulecolor{black} 
\begin{table}
\setlength\extrarowheight{3pt}
\begin{tabular}
  { 
  | M{15.5cm}
  |} 
  \hline 
  \textbf{Hecho reportado}  \\ [10pt]
  \hline 
   El 04 de octubre de 2023, del monitoreo permanente que realiza el Indecopi a través de los canales a los que tiene acceso, esta Dirección tomó conocimiento a través de una alerta vía correo electrónico que en las redes sociales se reportaba que la empresa promotora del evento  "Emilia Mernes" no estaría efectuando la devolución correspondiente al valor de las entradas adquiridas por los consumidores. Dicho evento estaba programado a realizarse inicialmente el 25 de noviembre de 2022 en el local denominado "Arena Perú", siendo reprogramado en más de una ocasión y por último cancelado. En ese sentido, el hecho de no devolver el monto de lo abonado por las entradas adquiridas para el referido evento, pese a que éste no se realizó en los términos inicialmente publicitados (cambio de fecha y local) siendo finalmente cancelado, podría configurar una presunta infracción al artículo 97 del Código de Protección y Defensa del Consumidor por parte de la empresa FANS & MUSIC ENTERTAIMENT S.A.C., identificada con número de RUC 20609462940, en su calidad de promotor del evento denominado "Emilia Mernes" (en adelante la empresa promotora). \\ [10pt]
  \hline
  \end{tabular}
\end{table}

\section{INDAGACIÓN ADICIONAL}\label{indagaciuxf3n-adicional}

Se realizaron acciones indagatorias: \setlength{\arrayrulewidth}{0.35mm}
\arrayrulecolor{black}

\begin{table} 
      \setlength\extrarowheight{5pt} 
      \begin{tabular}{| M{3.5cm} | M{3.5cm} | M{3.5cm} | M{3.5cm}|} 
      \hline 
      \textbf{Destinatario del requerimiento} & \textbf{Fecha del requerimiento} & \textbf{¿Fue reiterado?} & \textbf{¿Recibió respuesta?} \\ [10pt] 
      \hline\hline
Empresa promotora & 2023-02-08 & No & Sí\\ [10pt]
\hline
Teleticket & 2023-05-17 & Sí & Sí\\ [10pt]

      \hline\end{tabular} 
      \end{table}

\section{ANÁLISIS}\label{anuxe1lisis}

\subsection{HECHO  1 }\subsubsection{Evaluación liminar}

\begin{itemize}
\tightlist
\item
  La fiscalización del bien o servicio y hecho reportado recae en las
  Secretarías Técnicas de CC3, ILN o CCD.
\item
  El equipo NO cuenta con capacidad para programar acciones de
  fiscalización durante el año en curso.
\item
  El equipo NO cuenta con capacidad para programar acciones de
  fiscalización durante el TRIMESTRE en curso.

  \subsubsection{Análisis de riesgo}

  \begin{itemize}
  \tightlist
  \item
    El bien o servicio reportado SÍ ha sido incluido como parte del
    diagnóstico para materia de protección al consumidor y competencia
    desleal del PAF vigente.
  \end{itemize}
\item
  El puntaje de riesgo para el bien seleccionado es: 7900 y su nivel de
  riesgo es: Alto riesgo \setlength{\arrayrulewidth}{0.35mm}
  \arrayrulecolor{black}

  \begin{table}
  \setlength\extrarowheight{3pt}
  \begin{tabular}{| M{7.5cm} | M{7.5cm} |}
  \hline
  \textbf{Supuestos atenuantes} & \textbf{Sí(1)/No(0)} \\ [10pt]
  \hline
  \hline
   Subsanación voluntaria & NA\\ [10pt]
  \hline
   Medidas para evitar nuevos casos & NA\\ [10pt]
  \hline
   Colaboración & NA\\ [10pt]
  \hline
   Otro & NA\\ [10pt]
  \hline
  \end{tabular}
  \end{table}
  \setlength{\arrayrulewidth}{0.35mm}
  \arrayrulecolor{black}
  \begin{table}
  \setlength\extrarowheight{3pt}
  \begin{tabular}{| M{7.5cm} | M{7.5cm} |}
  \hline
  \textbf{Supuestos agravantes} & \textbf{Sí(1)/No(0)} \\ [10pt]
  \hline
  \hline
   Reincidencia en la conducta & 1\\ [10pt]
  \hline
   Daño a la salud & 0\\ [10pt]
  \hline
   No proporcionar información & 0\\ [10pt]
  \hline
   Otro & 0\\ [10pt]
  \hline
  \end{tabular}
  \end{table}
  \setlength{\arrayrulewidth}{0.35mm}
  \arrayrulecolor{black}
  \begin{table}
  \setlength\extrarowheight{3pt}
  \begin{tabular}{| M{15cm} |}
  \hline
  \textbf{Detalle la conducta del proveedor analizada para considerar los atenuantes y agravantes:} \\ [10pt]
  \hline
  \hline
   No se identificaron agravantes o atenuantes\\ [10pt]
  \hline
  \end{tabular}
  \end{table}
\item
  El puntaje de riesgo, considerando el valor del PAF y los atenuantes
  y/o agravantes identificados es: 9480 y su nivel de riesgo es: Muy
  alto riesgo

  \subsubsection{Selección de acciones}

  \begin{itemize}
  \tightlist
  \item
    La/s acción/es a realizar son: -Fiscalización clásica-
  \end{itemize}
\end{itemize}

\section{RESULTADOS DE LA
EVALUACIÓN}\label{resultados-de-la-evaluaciuxf3n}

\setlength{\arrayrulewidth}{0.35mm}
\arrayrulecolor{black}
\setlength\extrarowheight{5pt}
\begin{longtable}{| p{14.5cm} |}
\hline
\endhead
\subsection{Resultados del Hecho  1 }- La fiscalización del bien o servicio y hecho reportado recae en las Secretarías Técnicas de CC3, ILN o CCD.\\ [10pt]
- El equipo NO cuenta con capacidad para programar acciones de fiscalización durante el año en curso.\\ [10pt]
- El equipo NO cuenta con capacidad para programar acciones de fiscalización durante el TRIMESTRE en curso.\\ [10pt]
\hline
\end{longtable}

\end{document}
